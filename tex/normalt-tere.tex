\chapter{Normált terek}
Eddig csak valós számokra alkalmaztuk az abszolútérték függvényt, amikor hibafogalmakról beszéltünk. Megeshet, hogy a keresett érték nem egy valós szám, hanem például egy mátrix vagy egy függvény vagy egy tetszőleges operátor. Ilyenkor nem tudjuk alkalmazni a szokásos abszolút érték függvényt, mert nem tudjuk, hogy mit jelent egy mátrix abszolútértéke.

Ennek érdekében bezetünk egy olyan teret, melynek elemeire lehet a kiterjesztett abszolútérték függvényt használni.

\section{Normált tér}
Ahhoz, hogy kiterjesszük az abszolútérték függvényt tekintsük a tulajdonságait, hogy mit kéne örökölnie egy tágabb hossz fogalomnak:
\begin{enumerate}
    \item $|x| \geq 0 \quad \forall x \in \mathbb{R}$ es $|x| = 0 \iff x = 0$
    \item $|\lambda x| = |\lambda|\cdot|x|$ (abszolút homogenitás)
    \item $|x + y| \leq |x| + |y| \quad \forall x,y \in \mathbb{R}$ (háromszög egyenlőtlenség)
\end{enumerate}

\begin{definition}
    Legyen $X$ tetszőleges vektortér, és $\| \cdot \|:X \to \mathbb{R}$ egy függvény a következő tulajdonságokkal:
    \begin{enumerate}
        \item $\| x \| \geq 0 \quad \forall x \in X$ es $\| x \| = 0 \iff x = 0_{X}$ ($X$ nullvektora)
        \item $\| \lambda x \| = \lvert \lambda \rvert \cdot \| x \| \quad \forall x \in X, \forall \lambda \in \mathbb{R}$
        \item $\| x + y \| \leq \| x \| + \| y \| \quad \forall x, y \in X$
    \end{enumerate}

Ekkor ezen $\| \cdot \|$ függvényt normának nevezzük és a normált tér (N.T.) a következő rendezett pár: $(X, \| \cdot \|)$.
\end{definition}

\begin{definition}
    Ha $(X, \| \cdot \|)$ Normált tér, akkor $x, y \in X$ elemek távolságán az $\| x-y \|$ számot értjuk.
\end{definition}

\begin{megj}
    Ezt a $\| x-y \|$ távolságot szokás a norma által indukált metrikának nevezni.
\end{megj}


\begin{pelda}
    Példák normákra és normált terekre:
\begin{enumerate}
    \item $X = \mathbb{R}$ és $\| \cdot \| = \lvert \cdot \rvert$ 
    
    \item $X = \mathbb{R}^{n}$ a következő normákkal:
    \begin{enumerate}[(i)]
        \item $\| x \|_{1} :=   \sum \lvert x_{j} \rvert$
        \item $\| x \|_{2} := \sqrt{ \sum \lvert x_{j} \rvert^{2} }$
        \item $\| x \|_{\infty} := \max \{ \lvert x_{j} \rvert \}$
        \item $\| x \|_{p} := \left( \sum \lvert x_{j} \rvert^{p} \right)^{1 / p}$
        
        Ha $p \to \infty$ akkor $\| x \|_{p} \to \| x \|_{\infty} \quad \forall x \in X$
    \end{enumerate}

    \item $X = C[a, b]$, azaz az $[a, b]$ intervallumon értelmezett folytonos függvények, a következő normákkal:
    \begin{enumerate}[(i)]
        \item $\| f \|_{\infty} := \max_{x \in[a, b]} \lvert f(x) \rvert$ 
	\item $\| f \|_{\int} := \int _{a}^{b} \lvert f(x) \rvert \, dx$
    \end{enumerate}
		
\end{enumerate}
\end{pelda}

\section{Fontos fogalmak normált terekben}
Most hogy már kiterjesztettük a hossz fogalmát normált terekre, így képesek vagyunk az előző fejezetekben bezetett fogalmakat analóg módon megfogalmazni a tér normájával.
\begin{enumerate}
    \item \textbf{Hibafogalmak} \\
    Legyen $(X, \| \cdot \|)$ egy tetszőleges Normált tér és $a, \tilde{a} \in X$. Ekkor
    \begin{itemize}
        \item $\tilde{a}$ abszolút hibája: $a - \tilde{a} \in X$ 
        \item $\tilde{a}$ abszolút hibakorlátja: $\Delta_{a} \in \mathbb{R}$ szám, melyre $\| a- \tilde{a} \| \leq \Delta_{a}$ 
        \item $\tilde{a}$ relatív hibája: $\frac{a - \tilde{a}}{ \| \tilde{a} \|} \in X$
        \item $\tilde{a}$ relatív hibakorlátja: $\frac{\| a- \tilde{a} \|}{ \| \tilde{a} \|} \leq \delta_{a}\in\mathbb{R}$
    \end{itemize}

    
    \item \textbf{Konvergencia}
    \begin{definition}
        Azt mondjuk, hogy az $(x_{n}) \subset X$ sorozat konvergens, ha $\exists x \in X$, melyre $\| x_{n}- x \| \to 0$ ha $n \to \infty$.
    \end{definition}
\end{enumerate}

\section{Mátrixnormák}
Tudjuk, hogy az $\mathbb{R}^{n \times n}$-beli mátrixok a rajta értelmezett $+$ (összeadás) és $\lambda$-val való szorzás műveletekkel vektorteret alkotnak.
\begin{kerdes}
    Hogyan definiálható ezen a vektortéren norma?
\end{kerdes}

\begin{definition}
    Legyen $\| \cdot \| _\mathbb{R^{n}}$ egy $\mathbb{R}^{n}$ -beli vektornorma. Ekkor az $A \in \mathbb{R}^{n \times n}$ mátrix ezen vektornorma által indukált mátrixnormáján a következő számot értjük: 
    \begin{equation*}
\| A \| := \sup_{x \in \mathbb{R}^{n} \; x \neq 0} \frac{\| Ax \| _{\mathbb{R}^{n}}}{\| x \| _{\mathbb{R}^{n}}}
    \end{equation*}
\end{definition}
Magyarázó jelentések a definícióhoz:
\begin{itemize}
    \item $\| Ax \|_{\mathbb{R}^{n}}$ - az $Ax$ vektor "hossza"
    \item $\frac{\| Ax \| _{\mathbb{R}^{n}}}{\| x \| _{\mathbb{R}^{n}}}$ - hányszorosára nyújtotta az $A$ mátrix az $x$ vektort
    \item $\sup_{x \in \mathbb{R}^{n} \; x \neq 0} \frac{\| Ax \| _{\mathbb{R}^{n}}}{\| x \| _{\mathbb{R}^{n}}}$ - lehetséges legnagyobb megnyújtásnak az értéke
\end{itemize}

\begin{pelda}
    Tekintsük pár mátrixnak pár mátrixnormáját.
\end{pelda}
\begin{enumerate}
    \item \begin{equation*}
        \| I \| = \sup_{x \in \mathbb{R}^{n}, x \neq 0} \frac{\| Ix \| _{\mathbb{R}^{n}}}{\| x \| _{\mathbb{R}^{n}}} = \sup_{x \in \mathbb{R}^{n}, x \neq 0} \frac{\| x \| }{\| x \| } = \sup 1 = 1
    \end{equation*}
    Tehát bármelyik $\mathbb{R}^{n}$-beli norma által indukált mátrixnormában az identitás mátrix normája $1$, azaz $\| I \| = 1$.

    \item A sup-norma kiszámítása a tanult vektornormák esetén:
        Ha $\| \cdot \|_{\mathbb{R}^{n}} = \| \cdot \|_{1}$, akkor:
        \begin{equation*}
            \| A \|  = \| A \| _{1} = \max_{j \in \{ 1, \dots, n \}} \sum_{i = 1} ^{n} \lvert a_{ij} \rvert 
        \end{equation*}
        max oszlopösszeg! \\
        Például:
        \begin{equation*}
            \begin{bmatrix}
            -2 && 1 \\
            0 && 3
            \end{bmatrix}
            \implies \| A \|_{1} =  \max \{ \lvert -2 \rvert + \lvert 0 \rvert, \lvert 1 \rvert + \lvert 3 \rvert  \} = 3
        \end{equation*}

    \item Ha $\| \cdot \| = \| \cdot \|_{2}$, akkor:
    \begin{equation*}
        \| A \| = \| A \|_{2} = \sqrt{ \lambda_{\max} (A^{T}A) }
    \end{equation*}
    ahol $\lambda_{\max}$ a legnagyobb sajátértéket jelöli.
    Ezt a normát szokás \textit{spektrálnormának} nevezni, mert a sajátértékek halmazát \textit{spektrál}-nak nevezik.

    \item Ha $\| \cdot \| = \| \cdot \|_{\infty}$, akkor:
    \begin{equation*}
        \| A \| = \| A \|_{\infty} = \max_{i \in \{ 1, \dots, n \}} \sum_{j=1}^{n} \lvert a_{ij} \rvert 
    \end{equation*}
    max sorosszeg!\\
    Például:

    \begin{equation*}
        \begin{bmatrix}
        -2 & 1 \\
        0 & 3
        \end{bmatrix}
        \implies \| A \| _{\infty} = \max \{ \lvert -2 \rvert + \lvert 1 \rvert , \lvert 0 \rvert + \lvert 3 \rvert  \} = 3
    \end{equation*}
\end{enumerate}

\begin{allitas}\label{all:1}
    Az indukált mátrix normákra igazak a következők:
    \begin{enumerate}
        \item $\| Ax \| \leq \| A \| \cdot \| x \| \quad \forall A \in \mathbb{R}^{n \times n}$, $\forall x \in \mathbb{R}^{n}$.

        \item $\| I \| = 1$ (láttuk).

        \item $\| A \cdot B \| \leq \| A \| \cdot \| B \| \quad \forall A, B \in \mathbb{R}^{n\times n}$ (szub multiplikativitás).
    \end{enumerate}
\end{allitas}

\begin{megj}
    Vannak egyéb, nem indukált, mátrix normák. például:
    \begin{enumerate}
        \item $\| A \|' = \max_{i, j} \lvert a_{ij} \rvert$ (maximális elem)
        \item $\| A \|'' = \sum_{i, j = 1} ^{n} \lvert a_{ij} \rvert$ (elemösszeg)
        \item $\| A \|_{F} = \sqrt{ \sum_{i, j = n} ^{n} a_{ij}^{2} }$ (Frobenius norma)
    \end{enumerate}
    Ezekre a nem indukált mátrix normákra nem feltétlenül teljesülnek a \ref{all:1}-beli tulajdonságok.
\end{megj}

\section{Kondíciószám}
Az előbb meggondoltuk, hogy egy lineáris egyenletrendszernek, $Ax = b$-nek, az $A$ együtthatómátrixának egy elemét kicsit pertulbálva a megoldás drasztikusan változhat. Célunk, hogy megfogalmazzuk, hogy mennyire változhat a megoldás kis perturbációra.

A továbbiakban a következő egyenletrendszerrel fogunk foglalkozni.
\begin{equation}\label{eq:1}
    Ax = b 
\end{equation}
Ahol $A \in \mathbb{R}^{n \times n}, \; \det A \neq 0, \; b \in \mathbb{R}^{n}$

Tegyük fel, hogy  $b$ helyett a pertulbált $\tilde{b}$ van a jobb oldalon:
\begin{equation*}
    A\tilde{x} = \tilde{b}
\end{equation*}

Jelölje:
\begin{align*}
    \Delta x & = x- \tilde{x} \implies \tilde{x} = x - \Delta x \\
    \Delta b & = b - \tilde{b} \implies \tilde{b} = b - \Delta b
\end{align*}
Ekkor:
\begin{align*}
A\tilde{x} & = \tilde{b} \\
A(x - \Delta x) & = b - \Delta b \\
Ax - A\Delta x & = b - \Delta b \\ 
A\Delta x & = \Delta b \\
\Delta x & = A^{-1}\Delta b
\end{align*}

Nézzük $\| \Delta x \|$-át valamelyik $\mathbb{R}^{n}$-beli normában:
\begin{equation*}
    \| \Delta x \|  = \| A^{-1}\Delta b \| \leq \| A^{-1} \|  \cdot \| \Delta b \| 
\end{equation*}
Most alkalmazzük a \ref{eq:1}-es egyenletrendszerre a normát.
\begin{equation*}
    b = Ax
\end{equation*}
\begin{align*}
    \| b \|  = \| Ax \| & \leq \| A \| \cdot \| x \| \\
    \frac{1}{\| x \| } & \leq \| A \|  \cdot \frac{1}{\| b \| } \\
    \implies \frac{\| \Delta x \| }{\| x \| } & \leq \| A^{-1} \|  \cdot \| A \|  \cdot \frac{\| \Delta b \| }{\| b \| }
\end{align*}
Tehét azt kaptuk, hogy minél nagyobb $\| A^{-1} \|  \cdot \| A \|$ annál pontatlanabb a becslés.

\begin{definition}
    Az $\| A^{-1} \|  \cdot \| A \|$ számat az $A$ mátrix kondíció számának nevezzük és $\operatorname{cond}(A)$-val jelöljük.
\end{definition}

\begin{definition}
    Azt mondjuk, hogy a \ref{eq:1}-es egyenletrendszer rosszul kondícionált, ha $\operatorname{cond}(A) \gg 1$.
\end{definition}

\begin{pelda}
    Nezzük meg a már említett példának a kondíció számát.
    \begin{equation*}
    A = \begin{bmatrix}
        1 && 1  \\
        1 && 1.01
    \end{bmatrix}
    \end{equation*}
    Alkalmazzuk az $\| \cdot \|_{1}$ által indukált mátrix normát.
    \begin{equation*}
        \| A \| _{1} = \max \{1 + 1, \; 1 + 1.01\} = 2.01
    \end{equation*}
    \begin{equation*}
        A^{-1} =
        \begin{bmatrix}
            101 && -100 \\
            -100 && 100
        \end{bmatrix}
        \implies \| A^{-1} \| _{1} = \max \{101 + 100, \; 100 + 100 \} = 201
    \end{equation*}
    \begin{equation*}
        \operatorname{cond}(A) = 201 \cdot 2.01 = 404.01 \gg 1
    \end{equation*}
    Tehát valóban rosszul kondicionált volt az egyenlet rendszer.

\end{pelda}


