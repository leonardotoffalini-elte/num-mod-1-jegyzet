\chapter{Általános algebrai egyenletek megoldása}

Ebben a fejezetben egyismeretlenes valós egyenletekkel foglalkozunk. Egy ilyen egyenlet mindig felírható a következő alakban:
\begin{equation}\label{eq:7}
    f(x) = 0
\end{equation}
ahol $f: \mathbb{R} \to \mathbb{R}$ függvény.

Ezzel \ref{eq:7}-nek a megoldása ugyanaz mint $f$ zérushelye. Ezt keressük a továbbiakban!

\section{Gyökök stabilitása}
\begin{kerdes}
    Mennyire érzékeny a megoldás $f$ kis megvaltoztatasara?
\end{kerdes}

Tegyük fel, hogy \ref{eq:7} helyett az 
\begin{equation}\label{eq:8}
    \tilde{f}(x) = 0
\end{equation}
Egyenletet oldjuk meg, és tegyük fel, hogy \ref{eq:7}-nek és \ref{eq:8}-nek is $\exists!$ megoldása, melyek $x^{*}$ illetve $\tilde{x}^{*}$ rendre.

A következő legyen a mérőszámunk az eltérésre:
\begin{equation*}
    \lvert x^{*} - \tilde{x}^{*} \rvert  \leq \; \text{?}
\end{equation*}
Ha $f$ és $\tilde{f}$ csak \textit{kicsit} tér el egymástól, akkor legfeljebb mennyire tér el $x^*$ és $\tilde{x}^*$?
Mérje $\max \limits_{[a, b]} \lvert f - \tilde{f} \rvert$ az $f$ és $\tilde{f}$ eltérését.

Tegyük fel, hogy $f \in C[a, b] \cap D(a, b)$ 

\textit{Ismétlés:} (Lagrange-középérték tétél)
Tegyük fel, hogy $f \in C[a, b] \cap D(a, b)$. Ekkor $\exists c \in (a, b)$ úgy, hogy
\begin{equation*}
    f'(c) = \frac{f(b) - f(a)}{b - a}
\end{equation*}

Továbbá tegyük fel, hogy $x^{*}$ és $\tilde{x}^{*} \in [a, b]$, és $\max \limits_{[a, b]} \lvert f - \tilde{f} \rvert < \varepsilon$.
Alkalmazzuk a Lagrange-középérték tételt az $[x^{*}, \tilde{x}^{*}]$ intervallumon (feltéve, hogy $x^{*} < \tilde{x}^{*}$):
\begin{equation*}
    \exists c \in (x^{*}, \tilde{x}^{*}): \quad f(\tilde{x}^{*}) - f(x^{*}) = f'(c)(\tilde{x}^{*} - x^{*})
\end{equation*}
Tegyük fel, hogy $f'(x) \neq 0 \quad \forall x \in (x^{*}, \tilde{x}^{*})$.
\begin{equation*}
    \iff \lvert \tilde{x}^{*} - x^{*} \rvert = \left\lvert  \frac{f(\tilde{x}^{*})}{f'(c)}  \right\rvert  = \frac{\lvert f(\tilde{x}^{*}) - \tilde{f}(\tilde{x}^{*}) \rvert}{\lvert f'(c) \rvert } < \frac{\varepsilon}{\min \limits_{[a, b]} \lvert f' \rvert  }
\end{equation*}

\begin{definition}
    Az $M : = \frac{1}{\min \limits_{[a, b]} \lvert f' \rvert  }$ számot a \ref{eq:7} egyenlet \textit{kondicionáltsági számának} nevezzük.
\end{definition}

Tehát ha $\max \limits_{[a, b]}\lvert f - \tilde{f} \rvert < \varepsilon$, akkor $\lvert  \tilde{x}^{*} - x^{*} \rvert < M \cdot \varepsilon$.


\section{Konvergencia sebesség}
Tegyük fel, hogy $\lim_{ k \to \infty }x_{k} = x^{*}$, és legyen $e_{k} := x_{k} - x^{*}$. ($\lim_{ k \to \infty } e_{k} = 0$ vagy $\lim_{ k \to \infty }\lvert e_{k} \rvert = 0$)

\begin{definition}
    Azt mondjuk, hogy az $(x_{k})$ sorozat konvergencia rendje $p \geq 1$, ha
    \begin{equation*}
        \lim_{ k \to \infty } \frac{\log \lvert e_{k} \rvert }{\log \lvert e_{k-1} \rvert } = p
    \end{equation*}
    \begin{itemize}
        \item Ha $p = 1$, akkor lineáris vagy elsőrendű konvergenciáról beszélunk.
        \item Ha $p = 2$, akkor másodrendű vagy kvadratikus konvergenciáról beszélünk.
    \end{itemize}
\end{definition}

\begin{pelda} Elsőrendű és másodrendű konvergens sorozatok hibatagjainak lecsengésére példák.
    
\textbf{Elsőrendű:}
\begin{center}
\begin{tabular}{||c c c||} 
 \hline
  & $\lvert e_k \rvert$ & $\frac{\log \lvert e_{k} \rvert}{\log \lvert e_{k-1} \rvert}$ \\ [0.5ex] 
 \hline\hline
 $k = 1$ & $10^{-3}$ & N/A \\ 
 \hline
 $k = 2$ & $10^{-4}$ & $1.33$ \\
 \hline
 $k = 3$ & $10^{-5}$ & $1.25$ \\ [1ex] 
 \hline
\end{tabular}
\end{center}

\textbf{Másodrendű:}
\begin{center}
\begin{tabular}{||c c c||} 
 \hline
  & $\lvert e_k \rvert$ & $\frac{\log \lvert e_{k} \rvert}{\log \lvert e_{k-1} \rvert}$ \\ [0.5ex] 
 \hline\hline
 $k = 1$ & $10^{-3}$ & N/A \\ 
 \hline
 $k = 2$ & $10^{-6}$ & $2$ \\
 \hline
 $k = 3$ & $10^{-12}$ & $2$ \\ [1ex] 
 \hline
\end{tabular}
\end{center}
\end{pelda}


\begin{allitas}
    Tegyük fel, hogy $\lvert e_{k} \rvert = c_{k} \cdot \lvert e_{k-1} \rvert, \quad k = 1, 2, \dots$ ahol $0 < \underline{c} \leq c_{k} \leq \overline{c} < 1$. Valamilyen $\underline{c}$ és $\overline{c}$ konstansokra. \\
    Ekkor $x_{k} \to x^{*}$ monoton módon és elsőrendben.
\end{allitas}

\textit{Bizonyítás:}
\textit{Monotonan}, mivel $0 < c_{k} < 1 \implies \lvert e_{k} \rvert < \lvert e_{k-1} \rvert$ $\quad \forall k = 1, 2, \dots$ $\implies (\lvert e_{k} \rvert)$ sorozat monoton csökkenő.

\textit{Konvergál}, mivel $\lvert e_{k} \rvert = c_{k} \cdot \lvert e_{k_{1}} \rvert \leq \overline c \cdot \lvert e_{k-1} \rvert \leq \overline c \cdot \overline c \cdot \lvert e_{k-2} \rvert \leq \dots \leq \overline c^{k} \cdot \lvert e_{0} \rvert$.  Mivel $\overline c < 1$ ezért tényleg $\lim_{ k \to \infty }\lvert e_{k} \rvert = 0$.

A feltételben lévő egyenletnek mindkét oldalán logaritmust véve:
\begin{equation*}
    \log \lvert e_{k} \rvert  = \log c_{k} + \log \lvert e_{k-1} \rvert 
\end{equation*}
\begin{equation*}
    \implies \frac{\log \lvert  e_{k} \rvert }{\log \lvert e_{k-1} \rvert } = \frac{\log c_{k}}{\log \lvert e_{k-1} \rvert } + 1
\end{equation*}
Lltszik, hogy $\log \lvert e_{k-1} \rvert \to - \infty$. Mostmár elegendő lenne belátni, hogy $\log c_{k}$ korlátos.
\begin{equation*}
    \log \underline c < \log c_{k} \leq 0
\end{equation*}
Tehát $\frac{\log c_{k}}{\log \lvert e_{k-1} \rvert } \to 0 \implies$ a jobb oldal $\to 1$ $\implies$ $p = 1$ a konvergencia rendje, azaz elsőrendű a konvergencia.

\begin{allitas}
    Tegyük fel, hogy $\lvert e_{k} \rvert = c_{k} \cdot \lvert e_{k-1} \rvert^{p} \quad k = 1, 2, \dots$ ahol $p > 1$ és $0 < \underline{c} \leq c_{k} \leq \overline c < + \infty$. Valamilyen $\underline{c}$ és $\overline c$ konstansokra. Továbbá $\overline{c}^{1/p-1} \cdot \lvert e_{0} \rvert < 1$. Ekkor $(x_{k})$ konvergens és a konvergencia rendje $p$.
\end{allitas}

\begin{megj}
    Az utóbbi feltétel azt jeletnti, hogy a konvergencia csak akkor következik, ha $x_{0}$ elég közel van $x^{*}$-hoz. Ugyanakkor $\overline c < + \infty$, és nem kell teljesülnie, hogy $\overline c < 1$.
\end{megj}

\section{Intervallum felezés}
Megoldandó feladat: $f(x) = 0$

Feltevések:
\begin{itemize}
    \item $f \in C[a, b]$
    \item $f(a)f(b) < 0$
\end{itemize}
Ekkor a Bolzano tétel szerint $\exists x^{*} \in (a, b)$,  ahol $f(x^{*}) = 0$. Miután a Bolzano tétel biztosítja nekünk a gyök létezését, keressük meg, hogy hol van ez a gyök.

Felépítünk egy intervallumsorozatot:
$I_{0} := [a, b]$
Felezzük meg ezt az intervallumot, legyen $c = \frac{a+b}{c}$
Ezután vizsgáljuk $f(c)$ előjelét:
\begin{itemize}
    \item $f(c) = 0$ ekkor készen is vagyunk mert találtunk egy gyököt.
    \item Ha $f(c) \neq 0$, akkor $I_{1} := [a, c]$ vagy $I_{1} := [c, b]$, azt az intebrallumot választuk melyben az inteballum szélein az $f$ értéke ellentétes előjelű.
\end{itemize}

Megfelezzük $I_{1}$-et és folytatjuk az eljárást. Tehát megint megnézzük az intervallum felét és választjuk azt a felet, melyben az intervallum szélein az $f$ értéke ellentétes előjelű.

Az iteráció során mindig marad gyök az aktuálisa vizsgált intervallumban és mindig feleződik az intervallum hossza.

Látszik, hogy nem mindig fogunk olyan esetre találni, ahol $f(c) = 0$ ls pontosan megtaláltuk a függvény gyökét, például $f(x) = x - \sqrt{ 2 }$ függvénynek irracionális a gyöke de az iteráció során csak racionális pontokat vizsgálunk.

Tehát érdemes megbeszélni, hogy milyen pontossággal szeretnénk közelíteni a gyököt és mikor álljuk le.

Folytassuk addig az iterációt ameddig az aktuálisan vizsgált intervallum hossza nem éri el az előírt $\varepsilon > 0$ pontosságot.
Ekkor leállunk és válasszuk az aktuálisan vizsgált intervallum bármelyik pontját közelítő megoldásnak, mert az intervallumban minden pont legfeljebb $\varepsilon$ távolságra lesz a valós gyöktől.

Meg lehet mondani előre, hogy hány iteráció után kell majd leállnunk?

Jelölés: $\operatorname{diam} (I_{k}) := I_{k} \text{ hossza}$

$\operatorname{diam}(I_{k}) = \frac{b- a}{2^{k}} < \varepsilon$ ebből következik, hogy $k > \frac{\log\left( \frac{b-a}{\varepsilon} \right)}{\log(2)}$
Észrevétel: A lépésszám teljesen független az $f$ függvénytől, de hát miért is függne, mert mindig csak intervallumokkal dolgozunk és az $f$ függvényt csak a következő intervallum kiválasztására használjuk, ami lehetne akár egy pénzérme dobás is.

Érdemes lenne beszélni még a konvergencia sebességéről.

\begin{equation*}
    \lvert x_{k} - x^{*} \rvert  \leq \operatorname{diam}(I_{k})
\end{equation*}
ezen felsőkorlátok sorozata linárisan konvergens, mert $\operatorname{diam}(I_{k})$ mindig feleződik és az előző fejezetben megbeszéltük, hogy ha a hibatag valahanyadrészére csökken akkor a konvergencia lineáris.
($c_{k}:= \frac{1}{2} \quad \forall k$, láasd első állítás múlt óráról)


\section{Egyszerű iteráció (fixpont-iteráció)}
Megoldandó feladat: $f(x) = 0$

Írjuk át a következő alakra:
\begin{equation*}
    \varphi(x) = x \qquad
\end{equation*}
ahol $\varphi:\mathbb{R} \to \mathbb{R}$ valamilyen függvény.
Ekkor $f$ gyöke pontosan a $\varphi$ fixpontja.

Érvényes a fixponttétel a következő változata:
\begin{tetel}
    Legyen $H \subset \mathbb{R}$ zárt halmaz, és $\varphi: H \to H$ kontrakció, tehát $\exists q \in (0, 1)$, melyre $\lvert \varphi(x) - \varphi(y) \rvert \leq q \cdot \lvert x - y \rvert \quad \forall x , y \in H$. Ekkor
    \begin{itemize}
        \item egyértelműen létezik $\varphi$-nek fixpontja, azaz $\exists ! x^{*}$ melyre $\varphi(x^{*}) = x^{*}$
        \item tetszőleges $x_{0} \in H$ kezdőpontot választva a kovetkező módon definiált sorozat konvergens és tart $x^{*}$-hoz
        \begin{equation*}
            x_{k+1} = \varphi(x_{k})
        \end{equation*}
        \item a következő módon tudjuk becsülni a konvergencia sebességét $\lvert x_{k} - x^{*} \rvert \leq \frac{q^{k}}{1 - q} \cdot \lvert x_{1} - x_{0} \rvert$
    \end{itemize}
\end{tetel}

\begin{kerdes}
    Mikor kontrakció $\varphi$?
\end{kerdes}

Vegyük észre, hogy valamilyen módon a $\varphi'$ abszolútértékétől függ, hogy kontrakció-e a $\varphi$.

\begin{allitas}
    Tegyük fel, hogy $\varphi \in C(I)$ és $\varphi\in D(\operatorname{int}(I))$ tehát folytonos az intervallumon és differenciálható a belsejében. Ha $\exists q \in [0, 1)$, amely mellett $\lvert \varphi'(x) \rvert \leq q \quad \forall x \in \operatorname{int}(I)$, akkor $\varphi$ kontrakció $I$-n a $q$ kontrakciószámmal.
\end{allitas}

\textit{Bizonyítás:} Legyen $x, y \in I$ két tetszőleges pont, $x < y$ 
Alkalmazzuk $\varphi$-ra $[x, y]$ intervallumon a Lagrange-középérték-tételt:
Létezik $c \in ( x, y)$ melyre
$$
\varphi (y) - \varphi(x) = \varphi'(c) \cdot (y - x)
$$

Vegyünk mindkét oldalt abszolút értéket:
$$
\lvert \varphi(x) - \varphi(y) \rvert  = \lvert \varphi'(c) \rvert \cdot \lvert x - y \rvert \leq q \cdot \lvert  x- y \rvert \quad \forall x, y \in I
$$
Az egyenlőtlenség a feltétel miatt áll.


\begin{pelda}
    $\varphi(x) = \frac{1}{2}\cos(x)$ kontrakció-e a $\left[ 0, \frac{\pi}{2} \right]$ intervallumon? És ha igen mi a $q$ kontrakciószám?
    $\lvert \varphi'(x) \rvert = \left\lvert  -\frac{1}{2}\sin x  \right\rvert \leq \frac{1}{2} < 1 \quad \forall x \in [0, \frac{\pi}{2}]$ (sőt $\forall x \in \mathbb{R}$)
    Tehát $\varphi$ kontrakció és $q = \frac{1}{2}$ jó választás kontrakciószámra.
\end{pelda}

\begin{kerdes}
    Mi a konvergencia rendje?
\end{kerdes}

\begin{allitas}
    Tegyük fel, hogy $\varphi \in C^{p}[a, b]$, azaz $p$-szer folytonosan deriválható, és $\varphi$ beleképez $[a, b]$-be és $\varphi$ kontrakció $[a, b]$-n. Ha az $x^{*}$ fixpontban a következő igazak:
    \begin{align*}
        \varphi'(x^{*}) & = 0 \\
        \varphi''(x^{*}) & = 0 \\
        \varphi'''(x^{*}) & = 0 \\
        \varphi ^{(4)}(x^{*}) & = 0 \\
        & \; \; \vdots \\
        \varphi ^{(p-1)}(x^{*}) & = 0 \\
        \varphi ^{(p)}(x^{*}) & \neq 0 
    \end{align*}
    
    Ekkor tetszőleges $x_{0} \in [a ,b]$ pontból indítva a fixpont iterációt $p$-ed rendben konvergesn.
\end{allitas}

\textit{Bizonyítás:}
A konvergenciát biztosítja a fixpont tétel, tehát elég a kovergencia rendjét belátni.
Írjuk fel $\varphi$-nek $x^{*}$ körüli $p-1$-ed fokú Taylor polinomjának a hibáját az $x_{k}$ pontban
$\exists \vartheta_{k}$ az $x^{*}$ és $x_{k}$ között
\begin{align*}
    \varphi(x_{k}) - T_{p-1}(\varphi(x_{k}), x^{*}) & = \frac{\varphi ^{(p)}(\vartheta_{k})}{p!} (x_{k} - x^{*})^{p} \\
    \varphi(x_{k}) + \varphi(x^{*}) + 0 + 0 + \dots + 0 & = \frac{\varphi ^{(p)}(\vartheta_{k})}{p!} (x_{k} - x^{*})^{p} \\
    x_{k+1} - x^{*} & = \frac{\varphi ^{(p)}(\vartheta_{k})}{p!} (x_{k} - x^{*})^{p}
\end{align*}
Vegyük ezt abszolút értékben és vizsgáljuk így a konvergencia rendjét
\begin{align*}
    \lvert x_{k+1} - x^{*} \rvert & = \left\lvert \frac{\varphi ^{(p)}(\vartheta_{k})}{p!} (x_{k} - x^{*})^{p} \right\rvert \\
    \lvert x_{k+1} - x^{*} \rvert & = \left\lvert \frac{\varphi ^{(p)}(\vartheta_{k})}{p!} \right\rvert \cdot \lvert e_{k} \rvert ^{p} \\
    \lvert x_{k+1} - x^{*} \rvert & = c_{k} \cdot \lvert e_{k} \rvert ^{p}
\end{align*}
Kell még: $0 < \underline{c} \leq c_{k} \leq \overline c < + \infty$
$\varphi \in C^{p}[a, b]$ és $\varphi ^{(p)}(x^{*}) \neq 0$ ekkor $\lvert \varphi ^{(p)}(x) \rvert$ $x^{*}$ egy kis környezetében is pozitív. Ha $k$ elég nagy, akkor mivel $\vartheta_{k}$ $x_{k}$ és $x^{*}$ között van
\begin{equation*}
    \left\lvert  \frac{\varphi ^{(p)}(\vartheta_{k})}{p!}  \right\rvert 
\end{equation*}
beszorítható két pozitív konstans közé.


\section{Newton módszer (érintő módszer)}
Megoldandó feladat: $f(x) = 0$

Alapötlet:
\begin{enumerate}
    \item Tegyük fel, hogy $f$ differenciálható
    \item Vegyünk fel egy tetszőleges $x_{0} \in D(f)$ kezdőpontot.
    \item Húzzuk itt meg $f$ érintőjét.
    \item Ennek $x$ tengellyel való metszéspontja legyen $x_{1}$
    \item Folytassuk $x_1$-el az iterációt
\end{enumerate}

Megfelelő feltételekkel $x_{1}, x_{2}, \dots \to x^{*}$

\begin{kerdes}
    Mindig működik ez az eljárás?
\end{kerdes}

A módszer képlete:
$x_{k}$-beli érintő: $x = f'(x_{k})(x - x_{k}) + f(x_{k})$
$x$ tengellyel metszésőpontja:
\begin{align*}
    0 & = f'(x_{k})(x - x_{k}) + f(x_{k}) \\
    -f(x_{k}) & = f'(x_{k})(x - x_{k}) \\
    -\frac{f(x_{k})}{f'(x_{k})} & = x - x_{k} \\
    x_{k+1} & = x_{k} - \frac{f(x_{k})}{f'(x_{k})}
\end{align*}
\begin{kerdes}
    Mit lehet mondani a Newton módszer konvergencia rendjéről?
\end{kerdes}

\begin{allitas}
    Tegyük fel, hogy az $x^{*}$ gyököt és az egész $(x_{k})$ sorozatot tartalmlazó valamely $I$  intervallumban $f \in C^{2}(I)$, továbbá $\exists m_{1},M_{1}, m_{2}, M_{2} > 0$ konstansok, amelyekkel 
    \begin{equation*}
        m_{1} \leq \lvert f'(x) \rvert  \leq M_{1}
    \end{equation*}
    és
    \begin{equation*}
        m_{2} \leq \lvert f''(x) \rvert  \leq M_{2}
    \end{equation*}
    Ekkor $\frac{M_{2}}{2m_{1}}\lvert e_{0} \rvert < 1$ esetén a Newton módszer másodrendben konvergens.
\end{allitas}

\textit{Bizonyítás:}
Írjuk fel az $f$ függvény $x_{k}$ körüli elsőfokú Taylor polinomjának hibáját az $x^{*}$ pontban!
\begin{equation*}
    f(x^{*}) - f(x_{k}) - f'(x_{k})(x^{*} - x_{k}) = \frac{f''(\vartheta_{k})}{2!}(x^{*} - x_{k})^{2}
\end{equation*}
ahol $\varphi_{k}$ valamely pont az $x_{k}$ és $x^{*}$ között. Utána osszunk $f'(x_{k})$-val mindkét oldalt
\begin{align*}
    0 - \frac{f(x_{k})}{f'(x_{k})} - (x^{*} - x_{k}) & = \frac{f''(\vartheta_{k})}{2f'(x_{k})}(x^{*} - x_{k})^{2} \\
    x_{k+1} - x^{*} & = \frac{f''(\vartheta_{k})}{2f'(x_{k})}(x^{*} - x_{k})^{2} \\
    \lvert e_{k+1} \rvert & = \left\lvert  \frac{f''(\vartheta_{k})}{2f'(x_{k})}  \right\rvert \cdot \lvert e_{k} \rvert ^{2} \\
    \lvert e_{k+1} \rvert & = c_{k} \cdot \lvert e_{k} \rvert ^{2}
\end{align*}
Kell még, hogy $0 < \underline{c} \leq c_{k} \leq \overline c < + \infty$. A feltétel szerint
\begin{equation*}
    0 < \frac{m_{2}}{2M_{1}} \leq c_{k} \leq \frac{M_{2}}{2m_{1}} < + \infty
\end{equation*}
\begin{equation*}
    \frac{M_{2}^{1/2-1}}{2m_{1}} \cdot \lvert e_{0} \rvert  = \frac{M_{2}}{2m_{1}} < 1
\end{equation*}
esetén másodrendű a konvergencia.



