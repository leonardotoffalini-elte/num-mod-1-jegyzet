\chapter{Általános algebrai egyenletek megoldása}

Ebben a fejezetben egyismeretlenes valós egyenletekkel foglalkozunk. Egy ilyen egyenlet mindig felírható a következő alakban:
\begin{equation}\label{eq:7}
    f(x) = 0
\end{equation}
ahol $f: \mathbb{R} \to \mathbb{R}$ függvény.

Ezzel \ref{eq:7}-nek a megoldása ugyanaz mint $f$ zérushelye. Ezt keressük a továbbiakban!

\section{Gyökök stabilitása}
\begin{kerdes}
    Mennyire érzékeny a megoldás $f$ kis megvaltoztatasara?
\end{kerdes}

Tegyük fel, hogy \ref{eq:7} helyett az 
\begin{equation}\label{eq:8}
    \tilde{f}(x) = 0
\end{equation}
Egyenletet oldjuk meg, és tegyük fel, hogy \ref{eq:7}-nek és \ref{eq:8}-nek is $\exists!$ megoldása, melyek $x^{*}$ illetve $\tilde{x}^{*}$ rendre.

A következő legyen a mérőszámunk az eltérésre:
\begin{equation*}
    \lvert x^{*} - \tilde{x}^{*} \rvert  \leq \; \text{?}
\end{equation*}
Ha $f$ és $\tilde{f}$ csak \textit{kicsit} tér el egymástól, akkor legfeljebb mennyire tér el $x^*$ és $\tilde{x}^*$?
Mérje $\max \limits_{[a, b]} \lvert f - \tilde{f} \rvert$ az $f$ és $\tilde{f}$ eltérését.

Tegyük fel, hogy $f \in C[a, b] \cap D(a, b)$ 

\textit{Ismétlés:} (Lagrange-középérték tétél)
Tegyük fel, hogy $f \in C[a, b] \cap D(a, b)$. Ekkor $\exists c \in (a, b)$ úgy, hogy
\begin{equation*}
    f'(c) = \frac{f(b) - f(a)}{b - a}
\end{equation*}

Továbbá tegyük fel, hogy $x^{*}$ és $\tilde{x}^{*} \in [a, b]$, és $\max \limits_{[a, b]} \lvert f - \tilde{f} \rvert < \varepsilon$.
Alkalmazzuk a Lagrange-középérték tételt az $[x^{*}, \tilde{x}^{*}]$ intervallumon (feltéve, hogy $x^{*} < \tilde{x}^{*}$):
\begin{equation*}
    \exists c \in (x^{*}, \tilde{x}^{*}): \quad f(\tilde{x}^{*}) - f(x^{*}) = f'(c)(\tilde{x}^{*} - x^{*})
\end{equation*}
Tegyük fel, hogy $f'(x) \neq 0 \quad \forall x \in (x^{*}, \tilde{x}^{*})$.
\begin{equation*}
    \iff \lvert \tilde{x}^{*} - x^{*} \rvert = \left\lvert  \frac{f(\tilde{x}^{*})}{f'(c)}  \right\rvert  = \frac{\lvert f(\tilde{x}^{*}) - \tilde{f}(\tilde{x}^{*}) \rvert}{\lvert f'(c) \rvert } < \frac{\varepsilon}{\min \limits_{[a, b]} \lvert f' \rvert  }
\end{equation*}

\begin{definition}
    Az $M : = \frac{1}{\min \limits_{[a, b]} \lvert f' \rvert  }$ számot a \ref{eq:7} egyenlet \textit{kondicionáltsági számának} nevezzük.
\end{definition}

Tehát ha $\max \limits_{[a, b]}\lvert f - \tilde{f} \rvert < \varepsilon$, akkor $\lvert  \tilde{x}^{*} - x^{*} \rvert < M \cdot \varepsilon$.


\section{Konvergencia sebesség}
Tegyük fel, hogy $\lim_{ k \to \infty }x_{k} = x^{*}$, és legyen $e_{k} := x_{k} - x^{*}$. ($\lim_{ k \to \infty } e_{k} = 0$ vagy $\lim_{ k \to \infty }\lvert e_{k} \rvert = 0$)

\begin{definition}
    Azt mondjuk, hogy az $(x_{k})$ sorozat konvergencia rendje $p \geq 1$, ha
    \begin{equation*}
        \lim_{ k \to \infty } \frac{\log \lvert e_{k} \rvert }{\log \lvert e_{k-1} \rvert } = p
    \end{equation*}
    \begin{itemize}
        \item Ha $p = 1$, akkor lineáris vagy elsőrendű konvergenciáról beszélunk. \\
        \item Ha $p = 2$, akkor másodrendű vagy kvadratikus konvergenciáról beszélünk. \\
    \end{itemize}
\end{definition}

\begin{pelda} Elsőrendű és másodrendű konvergens sorozatok hibatagjainak lecsengésére példák.
    
\textbf{Elsőrendű:}
\begin{center}
\begin{tabular}{||c c c||} 
 \hline
  & $\lvert e_k \rvert$ & $\frac{\log \lvert e_{k} \rvert}{\log \lvert e_{k-1} \rvert}$ \\ [0.5ex] 
 \hline\hline
 $k = 1$ & $10^{-3}$ & N/A \\ 
 \hline
 $k = 2$ & $10^{-4}$ & $1.33$ \\
 \hline
 $k = 3$ & $10^{-5}$ & $1.25$ \\ [1ex] 
 \hline
\end{tabular}
\end{center}

\textbf{Másodrendű:}
\begin{center}
\begin{tabular}{||c c c||} 
 \hline
  & $\lvert e_k \rvert$ & $\frac{\log \lvert e_{k} \rvert}{\log \lvert e_{k-1} \rvert}$ \\ [0.5ex] 
 \hline\hline
 $k = 1$ & $10^{-3}$ & N/A \\ 
 \hline
 $k = 2$ & $10^{-6}$ & $2$ \\
 \hline
 $k = 3$ & $10^{-12}$ & $2$ \\ [1ex] 
 \hline
\end{tabular}
\end{center}
\end{pelda}


\begin{allitas}
    Tegyük fel, hogy $\lvert e_{k} \rvert = c_{k} \cdot \lvert e_{k-1} \rvert, \quad k = 1, 2, \dots$ ahol $0 < \underline{c} \leq c_{k} \leq \overline{c} < 1$. Valamilyen $\underline{c}$ és $\overline{c}$ konstansokra. \\
    Ekkor $x_{k} \to x^{*}$ monoton módon és elsőrendben.
\end{allitas}

\textit{Bizonyítás:}
\textit{Monotonan}, mivel $0 < c_{k} < 1 \implies \lvert e_{k} \rvert < \lvert e_{k-1} \rvert$ $\quad \forall k = 1, 2, \dots$ $\implies (\lvert e_{k} \rvert)$ sorozat monoton csökkenő.

\textit{Konvergál}, mivel $\lvert e_{k} \rvert = c_{k} \cdot \lvert e_{k_{1}} \rvert \leq \overline c \cdot \lvert e_{k-1} \rvert \leq \overline c \cdot \overline c \cdot \lvert e_{k-2} \rvert \leq \dots \leq \overline c^{k} \cdot \lvert e_{0} \rvert$.  Mivel $\overline c < 1$ ezért tényleg $\lim_{ k \to \infty }\lvert e_{k} \rvert = 0$.

A feltételben lévő egyenletnek mindkét oldalán logaritmust véve:
\begin{equation*}
    \log \lvert e_{k} \rvert  = \log c_{k} + \log \lvert e_{k-1} \rvert 
\end{equation*}
\begin{equation*}
    \implies \frac{\log \lvert  e_{k} \rvert }{\log \lvert e_{k-1} \rvert } = \frac{\log c_{k}}{\log \lvert e_{k-1} \rvert } + 1
\end{equation*}
Lltszik, hogy $\log \lvert e_{k-1} \rvert \to - \infty$. Mostmár elegendő lenne belátni, hogy $\log c_{k}$ korlátos.
\begin{equation*}
    \log \underline c < \log c_{k} \leq 0
\end{equation*}
Tehát $\frac{\log c_{k}}{\log \lvert e_{k-1} \rvert } \to 0 \implies$ a jobb oldal $\to 1$ $\implies$ $p = 1$ a konvergencia rendje, azaz elsőrendű a konvergencia.

\begin{allitas}
    Tegyük fel, hogy $\lvert e_{k} \rvert = c_{k} \cdot \lvert e_{k-1} \rvert^{p} \quad k = 1, 2, \dots$ ahol $p > 1$ és $0 < \underline{c} \leq c_{k} \leq \overline c < + \infty$. Valamilyen $\underline{c}$ és $\overline c$ konstansokra. Továbbá $\overline{c}^{1/p-1} \cdot \lvert e_{0} \rvert < 1$. Ekkor $(x_{k})$ konvergens és a konvergencia rendje $p$.
\end{allitas}

\begin{megj}
    Az utóbbi feltétel azt jeletnti, hogy a konvergencia csak akkor következik, ha $x_{0}$ elég közel van $x^{*}$-hoz. Ugyanakkor $\overline c < + \infty$, és nem kell teljesülnie, hogy $\overline c < 1$.
\end{megj}




